\documentclass{article}

\usepackage{ctex}
\usepackage{amsmath}
\usepackage[a4paper,right=2cm,left=2cm,top=1.5cm,bottom=1.5cm]{geometry}

\title{\zihao{1} 移动通信第二次作业}
\author{通信四班 220210404 张昕}
\date{\today}

\begin{document}
\maketitle
\zihao{4}
\begin{enumerate}
    \item 同频复用比例
          \begin{equation}
          Q=\frac{D}{R}
          \end{equation}
          由余弦定理
          \begin{equation}
            a^2+b^2-2\cos(C)ab=c^2
          \end{equation}可知
          \begin{equation}
            D=(\frac{\sqrt{3}}{2}Rj)^2+(\frac{\sqrt{3}}{2}Ri)^2+\frac{3}{4}R^2ij
          \end{equation}
          代入(1)得
          \begin{align}
            Q &=\frac{\sqrt{\frac{3}{4}R^{2}j^{2}+\frac{3}{4}R^{2}i^{2}+\frac{3}{4}R^{2}ij}}{0.5R}\\
              &=\sqrt{3i^2+3ij+3j^2}
          \end{align}
          又因为
          \begin{equation}
            N=i^2+ij+j^2
          \end{equation}
          得
          \begin{equation}
            Q=\sqrt{3N}
          \end{equation}
    \item \begin{enumerate}
        \item 总带宽为24MHz,两个30MHz信道提供双向信道,所以
              \begin{equation}
                S=\frac{24M}{2*30K}=400\text{个}
              \end{equation}
              所以
              \begin{equation}
                k=\frac{400}{4}=100\text{个}
              \end{equation}
        \item 信道利用率为90\%,所以总话务量
              \begin{equation}
                A=S*90\%
              \end{equation}
              又因为用户话务量
              \begin{equation}
                A_U=0.1Erlang
              \end{equation} 
              所以最大用户数目为
              \begin{equation}
                U=\frac{A}{A_u}=900
              \end{equation}
        \item 信道数目100,$A=90Erlang$,所以
              \begin{equation}
                G=0.03
              \end{equation}
        \item 每个扇区信道数目为33,呼阻率
              \begin{equation}
                G=0.03
              \end{equation}
              每个扇区话务量强度
              \begin{equation}
                A=25Erlang
              \end{equation}
              所以查表得每个扇区最大支持用户为250人,所以每小区支持用户数
              \begin{equation}
                N=750\text{人}
              \end{equation}
        \item 小区数
              \begin{equation}
                N_{cell}=\frac{50km*50KM}{5KM^2}=500
              \end{equation}
              所以用户量
              \begin{equation}
                U=N_{cell}*900=450000
              \end{equation}
        \item 用户量
              \begin{equation}
                U=N_{cell}*750=375000
              \end{equation}
            \end{enumerate}
    \item \begin{enumerate}
        \item 当$N=4$时,每个小区可用信道数为
              \begin{equation}
                C=\frac{300}{4}=75
              \end{equation}
              又因为
              \begin{equation}
                G=1\%
              \end{equation}
              查表得
              \begin{equation}
                A=60Erlang
              \end{equation}
              当$N=7$时,每个小区可用信道数为
              \begin{equation}
                C=\frac{300}{7}=43
              \end{equation}
              又因为
              \begin{equation}
                G=1\%
              \end{equation}
              查表得
              \begin{equation}
                A=31Erlang
              \end{equation}
              当$N=12$时,每个小区可用信道数为
              \begin{equation}
                C=\frac{300}{12}=25
              \end{equation}
              又因为
              \begin{equation}
                G=1\%
              \end{equation}
              查表得
              \begin{equation}
                A=16Erlang
              \end{equation}
        \item 当$N=4$时,
          \begin{equation}
            U=\frac{A}{A_u}*84=126000\text{个}
          \end{equation}    
          当$N=7$时,
          \begin{equation}
            U=\frac{A}{A_u}*84=65100\text{个}
          \end{equation}
          当$N=12$时,
          \begin{equation}
            U=\frac{A}{A_u}*84=33600\text{个}
          \end{equation}      
    \end{enumerate}
    \item 已知
          \begin{equation}
            P_r(dbm)=P_o(dbm)-10n\log(\frac{d}{d_o})
          \end{equation}
          设小区半径为R,且
          \begin{equation}
            Q=\frac{D}{R}=\sqrt{3N}
          \end{equation}
          所以
          \begin{equation}
            D=\sqrt{3N}R
          \end{equation}
          当$N=7$时
          \begin{equation}
            P_o(dbm)-10n\log(\frac{d}{d_o})<-100dbm
          \end{equation}
          解得
          \begin{equation}
            R>470m
          \end{equation}
          当$N=4$时
          \begin{equation}
            P_o(dbm)-10n\log(\frac{d}{d_o})<-100dbm
          \end{equation}
          解得
          \begin{equation}
            R>621m
          \end{equation}
    \item \begin{enumerate}
        \item 全双工\\
              每个信道带宽
              \begin{equation}
                \Delta=\frac{50M}{832}=60KHz
              \end{equation}
              60KHz的全双工信道分为前向(基站到用户)和反向(用户到基站)个30KHz,前者比后者高45MHz
        \item 发射频率
              \begin{equation}
                f=880.560M-45M=835.56MHz
              \end{equation}      
        \item A区,21个控制信道,395个话音信道\\
              B区,21个控制信道,395个话音信道
        \item N=18
        \item $Q=\dfrac{D}{R}=\sqrt{3N}$\\
              所以
              \begin{equation}
                D=\sqrt{3N}R
              \end{equation}
              当$N=4$时
              \begin{equation}
                D=4.58R
              \end{equation}
              当$N=7$时
              \begin{equation}
                D=3.46R
              \end{equation}
    \end{enumerate}
    \item \begin{enumerate}
        \item 基站费用为50万,MTSO费用150万,广告费用50万美元所以基站数量为
              \begin{equation}
                N=\frac{600-150-50}{50}=8\text{个}
              \end{equation} 
        \item 许可证覆盖面积140平方公里,每个小区面积
              \begin{equation}
                S=\frac{140km^2}{8}=17.5km^2
              \end{equation}      
              由六边形面积公式
              \begin{equation}
                S1=\frac{3\sqrt{3}R^2}{2}
              \end{equation}
              解得
              \begin{equation}
                R=2.6km
              \end{equation}
        \item $w=n*50*12+2n*50*12+4n*50*12+8n*50*12$
              \begin{equation}
                n=1112\text{名}
              \end{equation}
        \item $N=\dfrac{1112}{140km^2}=7.9\text{人}/km^2$
    \end{enumerate}
\end{enumerate}
\end{document}