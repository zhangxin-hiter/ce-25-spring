\documentclass{article}

\usepackage{ctex}
\usepackage[a4paper,left=2cm,right=2cm,top=2cm,bottom=2cm]{geometry}
\usepackage{tikz}
\usepackage{amsmath}

\usetikzlibrary{arrows.meta,positioning}

\title{\zihao{2} 通信网络第二次作业}
\author{220210404 通信四班 张昕}
\date{\today}

\begin{document}
\maketitle
\zihao{4}
\begin{enumerate}
    \item \begin{tikzpicture}[auto,thick,node distance=4cm,>=stealth]
        \tikzstyle{block}=[draw,rectangle,minimum height=2em,minimum width=3em]
        \node[block] (A) {源点};
        \node[block,right of =A] (B) {发送器};
        \node[block,right of =B] (C) {传输系统};
        \node[block,right of =C] (D) {接收器};
        \node[block,right of =D] (E) {终点};

        \draw [->] (A) -- node[above]{输入数据}(B);
        \draw [->] (B) -- node[above]{发送信号}(C);
        \draw [->] (C) -- node[above]{接收信号}(D);
        \draw [->] (D) -- node[above]{接收数据}(E);
    \end{tikzpicture}
    \begin{description}
        \item[源点] 产生要发送的数据
        \item[发送器] 将源点发送的数据比特流转化为信号
        \item[接收器] 接收传输系统发送的信号
        \item[终点] 接收接收器发送的数据,并转化为信息输出    
    \end{description}
    \item 信道带宽和信道中的信噪比 \\
          不能,实际情况中,信号功率不可能做的无限大 \\
          档信息传输速率低于极限信息传输速率时,我们就办法实现无差错传输,只是香农公式未给出  \\
          一个码元可能含有多个比特,所以比特每秒并不等价于码元每秒
    \item 由香农公式
          \begin{equation}
            C=W\log_2(1+\frac{S}{N})
          \end{equation}
          当$W=3100HZ$,$C=3600bit/s$时,解得$\frac{S}{N}=3131$ \\
          当$W=3100HZ$,$C=57600bit/s$时,解得$\frac{S}{N}=392040$ \\
          不能
    \item 由码分多址接入CDMA可知,不同站的码元序列向量正交
          \begin{equation}
            S \cdot T \equiv \frac{1}{m}\sum_{i=1}^{m}s_i*t_i
          \end{equation}
          同站码元序列向量内积为0
          \begin{equation}
            S \cdot S \equiv \frac{1}{m}\sum_{i=1}^{m}s_i*s_i
          \end{equation}
          反码则为-1\\
          令X=(-1,+1,-3,+1,-1,-3,+1,+1),有
          \begin{equation}
            A \cdot X =1
          \end{equation}
          \begin{equation}
            B \cdot X =-1
          \end{equation}
          \begin{equation}
            C \cdot X =1
          \end{equation}
          \begin{equation}
            D \cdot X =0
          \end{equation}
          所以,A和D发送数据为1,B为0,D没有发送数据
\end{enumerate}
\end{document}